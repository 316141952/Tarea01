\documentclass{book} %para darle formato de libro
\usepackage{graphicx} %pa' ponerle imágenes
\usepackage[spanish]{babel} %para que las cositas predeterminadas aparezcan en español
\usepackage{hyperref} %para ponerle links
%\usepackage{biblatex} % para ponerle bibliografía 
\usepackage[utf8]{inputenc} %pa' poner acentitos 
\graphicspath{{/home/Valeria/thc/Clases/Latex/Imagenes}}

\title{\Huge Tarea 1\\Programación\\Universidad Nacional Autónoma de México.\\Facultad de Ciencias.\\}
\author{\huge Valeria Ortiz Cervantes}
\date{\LARGE 27 de febrero del 2019}

\begin{document}
	
\maketitle
\includegraphics[scale=3]{/1.jpg}
\newpage

\tableofcontents

\chapter{\large Historia de la Computación.}
\section{Inicios de la programación.}
La computación surgió debido a que las personas querían hacer cálculos grandes en menos tiempo, querían crear algo que lo hiciera por ellos.\\La primera calculadora la inventó Blaise Pascal en 1642, era hijo de un recaudador de impuestos y buscaba la forma de reducir el trabajo de sumar grandes cantidades de números. Años después Leibnitz mejoró la máquina de Pascal, añadiéndole un cilindro escalonado, conocido ahora como rueda de Leibnitz para representar los dígitos del 1 al 9. En 1673 construyó su máquina calculadora, que era verdaderamente superior a la de Pascal, podía sumar, restar, multiplicar, dividir y obtener raíces.\\En 1823 Charles Babbage obtuvo una subvención del gobierno británico para crear una máquina de diferencias, motivado por el tedioso proceso de realizar tablas matemáticas. Mientras tanto Joseph Jacquard, quien era obrero en una fábrica de sedas de Lyon, introduce la idea de programar máquinas mediante el uso de tarjetas perforadas, creó un telar que utilizaba tarjetas perforadas para controlar de manera automática el diseño y los colores de los tejidos. \\Al saber sobre el telar programable de Jacquard, Babbage abandonó la máquina de diferencias y se dedicó a la llamada máquina analítica de Babbage, que puede considerarse el antecedente directo del ordenador actual. Ideada en 1835, no llegó a realizarse, probablemente por la incapacidad de la tecnología, meramente mecánica, de la época. La idea era que pudieras utilizar resultados previos, es decir, hacer cálculos iterativos. De esta manera se identificaban las etapas de una tarea informática como entrada, tratamiento y salida de datos asociadas a los distintos elementos de la máquina. \\La hija de Lord Byron, Ada Augusta Byron, condesa de Lovelace, quedó fascinada por la máquina analítica y colaboró en su diseño, aportando nuevas ideas y corrigiendo los errores del trabajo de Babbage. También construyó varios procedimientos para utilizar la máquina de Babbage en la resolución de varios problemas. Como consecuencia de sus aportaciones, Ada Lovelace se considera la primera programadora de la historia.
\section{Precedentes de los primeros computadores.}
Para el censo norteamericano de 1890, Herman Hollerith diseñó un sistema compuesto de una lectora eléctrica de tarjetas perforadoras, una clasificadora rudimentaria y una unidad tabuladora para realizar las sumas e imprimir los resultados. La máquina tabuladora fue capaz de concluir el recuento del censo de 1890 en menos de tres años. En 1911, Hollerith funda la Computing-Tabulating-Recording Machine Company, que posteriormente, reorganizada por Thomas J. Watson sería el preludio de la fundación de IBM. \\En 1936 Alan Turing especificó un ordenador teórico completamente abstracto que pudiera llevar a cabo cualquier cálculo realizable por un ser humano (la Máquina Universal de Turing). Aprovechó la oportunidad para dar vida a sus ideas mediante sus investigaciones sobre lo que generalmente se consideran los primeros ordenadores digitales electrónicos funcionales del mundo, desarrollados en Gran Bretaña durante la Segunda Guerra Mundial.
\section{Primeras computadoras analógicas.}
Con el desarrollo posterior de la electricidad aparecieron las llamadas computadoras electromecánicas, las cuales utilizaban solenoides e interruptores mecánicos operados eléctricamente. La primera de ellas se creó en 1944 y fue la llamada Mark I. Las instrucciones se cargaban por medio de cinta de papel con perforaciones, y los datos se proporcionaban en tarjetas también perforadas. Esta computadora tenía aproximadamente 15.5 m. de largo por 2.5 de altura, y multiplicaba dos números en aproximadamente 3 segundos. Tres años más tarde, la computadora Mark II. era capaz de llevar a cabo la misma operación en menos de un cuarto de segundo, 12 veces más rápido.
\section*{Primeras computadoras digitales.}
Mientras estas computadoras analógicas eran construidas, se gestaba un nuevo concepto de computadoras. Éstas eran las llamadas computadoras digitales, acerca de cuya paternidad existen gran cantidad de disputas. Sin embargo, en una batalla legal se atribuyó el derecho a llamarse "inventor de la computadora digital" a John V. Atanasoff. Su máquina de calcular, conocida como ABC (AtanasoffBerry Computer), fue creada en 1939, estaba basada en el uso de tubos de vacío y operaba en binario. 
El ABC no pretendía el cálculo universal, como el ENIAC, desarrollado para el ejército de los Estados Unidos por John Presper Eckert y John W. Mauchly, que utilizaba tubos de vacío con tecnología basada en diodos y triodos. Su velocidad de trabajo era mil veces superior a la de las máquinas electromecánicas y una hora de trabajo del ENIAC era equivalente a una semana del Mark I. El ENIAC ya incorporaba todos los conceptos modernos sobre el ordenador tales como la unidad central de proceso, una memoria y entrada y salida de datos. 

\chapter{Arquitecturas de computadoras.}
\section{Arquitectura de Von Neumann.}

\subsection{Unidad Aritmético-Lógica.}

\subsection{Unidad de Control.}

\subsection{Unidad de Memoria.}

\subsection{Dispositivos de entrada y salida.}
\begin{enumerate}
	\item Entrada :
	\item Salida :
\end{enumerate}

\section{Arquitectura de multiproceso}

\section{Arquitectura segmentada.}

\chapter{Algoritmo}
\section{Concepto.}

\section{Complejidad de un algoritmo.}

\section{Aplicaciones prácticas.}

\chapter{Paradigma orientado a objetos.}
\section{Ventajas.}
\begin{enumerate}
	\item
\end{enumerate}
\section{Desventajas.}
\begin{enumerate}
	\item
\end{enumerate}

\section{Lenguajes que usan el paradigma orientado a objetos}

\chapter{Conceptos básicos.}
\section{Ciclos.}
\section{Condicionales.}
\section{Tipos.}
\section{Estado.}
\section{Memoria.}
\section{Librería.}
\section{Documentación de programas.}
\section{Compilar.}
\section{Interpretar.}
\section{Máquina virtual.}
\section{Palabras reservadas.}

\end{document}